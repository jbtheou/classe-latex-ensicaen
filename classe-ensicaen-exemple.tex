\documentclass[times]{rapport-ensi}
\titre{Utilisation de la classe rapport-ensi}
\soustitre{Mon sous titre}
\specialite{Électronique et physique appliqué}
\annee{2\ieme}
\nomentreprise{ENSICAEN}
\adresseentreprise{6, bd Maréchal Juin \newline F - 14050 Caen Cedex 4}
\nom{Jean-Baptiste TH\'EOU}
\imageannee{img/imageannee-2010-2011.jpg}
\setlength{\widthentreprise}{5cm}
\logoentreprise{img/logoecole}
\logoecole{img/logoecole}
\docpdf


\begin{document}
 \fairetitre
 \tableofcontents
\chapter*{Introduction}
L'introduction est appelée à l'aide de la fonction chapter*
\chapter{Utilisation de la classe}
Pour utiliser cette classe, il suffit de copier rapport-ensi.cls dans le dossier contenant votre source .tex. 
Elle est inspirée des modèles pour Openoffice et Word, donnés pour la filière informatique (Disponible sur l'intranet
de l'ENSICAEN).

\section{Les différentes options}
Les différentes options\footnote{A définir avant le begin\{document\}} de cette classe sont :
\begin{itemize}
\item titre : Le titre du rapport (Ici \og{}Utilisation de la classe rapport-ensi\fg{} )
\item soustitre : Le sous titre du rapport (Ici \og{}Mon sous titre\fg{} )
\item specialite : Votre spécialité (Ici \og{}Électronique et physique appliquée\fg{} )
\item annee : Votre année (Ici \og{}2\ieme\fg{} )
\item nom : Votre nom :)
\item imageannee : L'image qui se trouve en bas de page
\item nomentreprise : Le nom de l'entreprise dans laquelle vous effectuez votre stage
\item adresseentreprise : L'adresse de l'entreprise\footnote{Pour avoir un retour à la ligne, utiliser la commande \newline}
\item logoentreprise : Le logo de votre entreprise
\item widthentreprise : Taille du logo entreprise (La largeur). A adapter au logo. Il faut la définir avec setlength
\end{itemize}
Il y a 3 autres options que normalement vous n'avez pas à modifier
\begin{itemize}
\item logoecole : Le logo de l'école (Par défaut, il s'appelle logoecole.jpg)
\item widthecole : La taille du logo. Ici la largeur est de 5 cm avec le logo fourni
\item adresseecole : L'adresse de l'école.
\item type : Le type de document. Par défaut, un rapport de stage.
\end{itemize}
\subsection{Option de classe}
Pour la classe rapport-ensi, il existe une option : times. 
Si vous mettez l'option times, c'est la police times qui est utilisée. 
Si vous souhaitez utiliser une police qui vous est propre, 
ou utiliser la police par défaut de la classe report, ne pas utiliser cette option. Dans cette exemple, l'option times est active.
\end{document}

 
\end{document}
